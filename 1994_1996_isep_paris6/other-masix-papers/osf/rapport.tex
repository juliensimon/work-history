% -*-latex-*-
%
%	Article pour workshop OSF
%

\documentstyle [11pt,a4,epsf,fancyheadings,rcnamed,pdareport,twoside] {article}

	\raggedbottom

%
% Les exemples de code
%

\newcommand {\code} [1]
	{\begin {quote}
		\tt
		\begin {tabbing}
			#1
		\end {tabbing}
	 \end {quote}
	}
\newcommand {\cindent} {\hspace* {10mm} \=}

%
% Les choses a faire
%

\newcommand {\todo} [1]
	{\begin {center}
		\Large {A faire: #1}
	 \end {center}
	 \addcontentsline {toc} {subsubsection} {A faire: #1}
	}

%
% Les changements a defaut de changebar
%

\newcommand {\changementdebut}
	{\marginpar {\rule [2pt] {10mm} {1mm} $\bigtriangledown$}}
\newcommand {\changementfin}
	{\marginpar {\rule {10mm} {1mm} $\bigtriangleup$}}

%
% Et hop, c'est parti !
%
\begin {document}

	\bibliographystyle {rcnamed}

%\title {The MASIX Multi-Server Operating System}
%\author {R\'emy Card \and \'Eric Commelin \and St\'ephane Dayras
%	 \and Franck M\'evel \\
%	 E-Mail: \{card,commelin,dayras,mevel\}@masi.ibp.fr \\ \\
%	Laboratoire MASI, Institut Blaise Pascal \\
%	Universit\'e Pierre et Marie Curie
%}
%\date { ~ }
%\maketitle

\begin {center}
{\Large The MASIX Multi-Server Operating System}

R\'emy Card, \'Eric Commelin, St\'ephane Dayras, Franck M\'evel \\
E-Mail: \{card,commelin,dayras,mevel\}@masi.ibp.fr

Laboratoire MASI --- Institut Blaise Pascal \\
Universit\'e Pierre et Marie Curie \\
4, place Jussieu --- 75252 Paris Cedex 05 \\

\end {center}

{\bf Abstract}

	Traditional operating systems have been designed as a monolithic
kernel implementing the whole set of the system services. This approach leads
to important maintenance end evolution problems. This paper discusses a new
design methodology. The operating system is implemented as a set of servers
using the services provided by a micro-kernel. Splitting up the operating
system into a set of servers leads to a modular and evolutive structure. The
system exposed in this paper is made up of three layers. The first one, the
host system, provides services which allow the dynamic inclusion of new servers
and different user interfaces. The second one, the Unix environment, contains
a set of servers which provide Unix abstractions and semantics. It is based
on new and optimized implementations, using the Mach micro-kernel facilities.
The third layer, the virtual system, is made up of a set of servers providing
distributed features. These servers are dynamically included in the system by
using the host system services. Numerous improvements included in the system
imply performance optimizations in spite of context switches and messages
exchanges needed to make different servers communicate with each others. The
resulting structure is innovative and easy to extend. This system is the basis
for future distributed operating system researches.

{\bf R\'esum\'e}

        Les syst\`emes d'exploitation ont traditionnellement \'et\'e con\c{c}us
sous forme d'un noyau monolithique offrant tous les services n\'ecessaires.
Cette approche pose de gros probl\`emes de maintenance et d'\'evolution du
syst\`eme. Cet article propose une m\'ethode originale de conception de
syst\`eme sous forme de serveurs utilisant les services offerts par un
micro-noyau. La d\'ecomposition du syst\`eme en serveurs permet d'obtenir une
structure modulaire et tr\`es \'evolutive. Le syst\`eme d\'ecrit dans cet
article est compos\'e de plusieurs couches. La premi\`ere, le syst\`eme
d'accueil, offre tous les services permettant d'inclure dynamiquement de
nouveaux services et des interfaces utilisateur diff\'erentes. La seconde,
l'environnement Unix, est constitu\'ee d'un ensemble de serveurs r\'ealisant
les abstractions et les s\'emantiques offertes par le syst\`eme Unix.
Cet environnement repose sur des impl\'ementations nouvelles et optimis\'ees
tenant compte des fonctionnalit\'es offertes par le micro-noyau Mach. La
troisi\`eme couche, le syst\`eme virtuel, est constitu\'ee d'un ensemble de
serveurs impl\'ementant des fonctionnalit\'es r\'eparties. Ces serveurs sont
inclus dynamiquement dans le syst\`eme en utilisant les possibilit\'es du
syst\`eme d'accueil. De nombreuses am\'eliorations incluses dans le syst\`eme
permettent d'optimiser ses performances malgr\'e les changements de contexte
et \'echanges de messages n\'ecessaires pour communiquer entre serveurs
diff\'erents. Le syst\`eme r\'esultant poss\`ede une structure novatrice,
facilement extensible et constitue la base de recherches futures dans le
domaine des syst\`emes d'exploitation r\'epartis.

\section* {Introduction}

	Masix is a new operating system, currently under development at the
MASI laboratory. Masix is a multi-environments operating system, i.e. it allows
the parallel execution of different environments. Thus, different users may
use different interfaces.

	Our system has been designed with evolution in mind and allows the
dynamic extension of its semantics while it is running. It is based on a
multi-servers model which allows each component of the system to be written
without interactions with the other ones. This also leads to a high level
of maintainability.

	Since Unix is a widely used system in research environments, Masix
offers a Unix compatible interface. This allows us to re-use the whole
set of Unix programs.

	Masix has also been designed to allow the optimal use of resources
in a networked environment. Thus, it includes distributed extensions that
optimize the use of a set of communicating stations.

	In this paper, we first present the principles that we have used to
design Masix. Then, we focus on the way the servers operate to build
the local system. Last, we describe the distributed extensions included in
the system.

\section {Design Principles}

\subsection {System Structure}

	Traditional operating systems have been designed as a single
monolithic program, called the kernel, which provides the whole set of
system calls to the user processes. In this scheme, the kernel runs in
protected mode and processes trap in the kernel to execute system calls.
Examples of such operating systems are Locus\cite {popek:walker:locus},
OSF/1 IK\cite {osf1}, Sprite\cite {ousterhout:sprite} and
Unix\cite {bach,leffler:bsd}.

	A current trend in operating system research consists in implementing
the system in user mode with a minimal kernel. This kernel, called a
microkernel, provides the resource management and some elementary services,
e.g. physical memory management, process management... Examples of current
microkernels include Amoeba\cite {mullender:amoeba},
Chorus\cite {rozier:chorus:overview}, Mach\cite {mach:foundation} and
V-Kernel\cite {cheriton:v}.

	The design of the Masix operating system\cite {card:these} has been
based on the microkernel approach. The reasons why we have chosen the
Mach microkernel are the following:
\begin {itemize}
\item it has been ported on numerous computers,
\item it is freely available from the Carnegie Mellon University,
\item it is at the basis of several research projects from which we can
benefit.
\end {itemize}

	Three kinds of operating systems have been built on top of Mach:
\begin {enumerate}
\item OSF/1 IK\cite {osf1} has been designed as a traditional monolithic
kernel. It uses Mach as a basis for its low-level layers.
\item Single-server systems, e.g. OSF/1 MK\cite {osf1mk}, implement Unix
semantics in a single task which relies on Mach for the low-level operations.
\item Multi-server systems, e.g. CMU's Unix Multi
Server\cite {mach:multiserv1}, contain several tasks, which cooperate to
provide Unix semantics.
\end {enumerate}

	We believe that it is worth designing an evolutive system in which
components can
be added in a dynamic way. Therefore, we have chosen to build Masix following
the multi-servers approach.

	Implementing the operating
system as a set of communicating user mode tasks offers many advantages:
\begin {itemize}
\item each server can be written and tested without interactions with the
other ones,
\item each server is smaller than a traditional kernel and, thus, easier
to maintain,
\item adding a server in the running system is easy if a dynamic service
management has been implemented.
\end {itemize}

	Nevertheless, the multi-servers approach has also its drawbacks:
\begin {itemize}
\item shared data are difficult to implement because each server has its
own address space,
\item the execution of system services often involves several servers and
implies message exchanges and context switches. This leads to lower
performances due to the context switch overhead.
\end {itemize}

	Lots of optimizations have been included in the design of the system
to minimize the number of servers involved in the realisation of a system call.
Moreover, we have reduced the number of system calls which need to interact
with a server.

	System calls are implemented as remote procedure calls: a process
sends a request message to a server and waits for the reply. The system
contains three kinds of servers:
\begin {enumerate}
\item interface servers, which receive requests from the user processes,
\item internal servers, which provide features needed by other servers,
\item generic servers, which are contained in the host system and which
provide services needed to run several environments.
\end {enumerate}
	
\subsection {Naming and Protection}

	In Masix, each server is responsible for providing a set of well-defined
abstractions and semantics. The combination of the whole set of servers
forms the complete operating system.

	Each server implements the operations that act on typed objects. The
server must be able to understand a fixed dialog protocol related to the
type of the objects. When a client process needs to act on an object, it sends
a request message to the server that manages the object. This message is sent
on a Mach port, which is used as the object identifier. The server holds the
reception right on this port.

	Accessing an object only requires to know what the object port and its
associated dialog protocol are. This
request is forwarded by the Mach microkernel to the server which manages the
object. This communication scheme allows several servers, which manage the
same objects types in different ways, to coexist in the system as long as
they implement the same dialog protocol. The client process only needs to
know an object port and its associated dialog protocol to act on the object.

	If a station running Masix mounts local and remote file
systems, for example, two file servers run in the system.
The first server implements I/O
operations on local files and the second one implements I/O operations on
remote file. When a process accedes to a file, it sends the same request
wether the
file is local or remote and this request is forwarded to the appropriate
server.

	Naming objects with Mach ports also allows a high level of protection.
To act on an object, a process needs to hold a send right on its port. This
send right is transmitted by the server which manages the object when the
process accedes to the object for the first time, e.g. when it creates or
opens the object. During this first access, the server checks that the calling
process is allowed to act on the object and returns the send right. If the
process is not allowed to accede to the object, the server does not return
any send right. Thus, if the server has denied access to the object, the
process cannot send any request since it does not hold any send right to the
port.


\subsection {Server Structure}

	In Masix, each server is a Mach task. This task contains several
threads which wait for requests and execute them. The threads use port sets
to wait for the requests: each object is named by a port and every object
port is contained in a single port set. Every thread waits for requests on
this port set. Requests are thus dynamically distributed between the
threads contained in the server.

	It is worth noting that this scheme implies synchronization between
the threads. As the threads accedes to the same data contained in the server's
address space, each thread needs to lock the data it uses to avoid
inconsistencies if it is preempted by the Mach scheduler.

\subsection {System Calls and Library Functions}

	In Masix, system calls are implemented as remote procedure calls and
imply messages passing and context switches. This scheme is much more expensive
than the traditional one in which processes trap into the kernel to make a
system call. Thus, to reduce the overhead, we have worked to reduce the number
of actual system calls. Many system calls have been moved as library functions
which are executed in the calling process address space. These calls do not
require context switching and are more efficient.

	Of course, some system calls cannot be moved as library routines:
\begin {itemize}
\item Some calls need a global knowledge of the system state. Implementing
them as library subprograms would require to make these informations available
to the calling process in its address space.
\item The parameters of some calls must be checked and authentified by the
system. To make the system safe, this check must be made by a separate
server. This way, an erroneous or malicious program cannot modify crucial
data.
\item Some calls modify data shared by several processes. Since each process
has its own distinct address space, the calling process cannot modify the
data itself. Thus, the data must be stored in a server address space and the
processes must send requests to the server to accede to them.
\end {itemize}

\subsection {Interactions between servers}

	Each server needs to maintain some informations relative to its
objects. Two servers often need to use the same information. In a monolithic
kernel, these informations are stored in common tables which are acceded to by
every component which needs to. In a multi-servers system, we cannot use global
tables since each server has its own address space. In Masix, each server
maintains a subset of the system informations. This subset contains only the
informations that the server needs to know in order to manage its objects.

	Sharing informations between servers is a real problem and we have
distinguished three cases:
\begin {enumerate}
\item Read-only informations, e.g. a process identifier, are duplicated in
each server which needs to know them. This way, each server can accede to the
informations in its address space.
\item Informations that are more often read than modified are replicated in
each server. When a server modifies an information, it sends its new value to
the other ones.
\item Informations that are more often modified than read are managed by a
server. Other servers must send requests to this server to accede to the data.
\end {enumerate}

\section {The Local System}

\subsection {The Administration and Configuration Server}

	Masix is a multi-environments system. It includes an environment
manager, which is responsible for:
\begin {itemize}
\item keeping track of the environments running in the system,
\item providing a way for the user processes to communicate with the
environments,
\item sharing the system resources between the environments.
\end {itemize}

	The Administration and Configuration Server (ACS) is the first server
to run when Masix is booted. It is in charge of maintaining the list of
the servers which run in the system. The ACS allows clients to dynamically
communicate with the servers. For this purpose, it offers a name service: it
holds the reception right on a global port known by every task and uses this
port for answering the name requests.

	When a server starts its execution, it registers itself with
the ACS by sending a message on this port. This message contains:
\begin {itemize}
\item a character string used as the name of the server,
\item a flag, which is true if the server is essential for the system, false
otherwise,
\item a send right on its request port.
\end {itemize}

	The ACS maintains the list of registered servers and can receive three
kinds of requests:
\begin {enumerate}
\item register requests, sent by servers when they start their execution,
\item client requests, sent by the clients when they need to know the request
port associated to a server,
\item unregister requests, sent by servers when they terminate their
execution.
\end {enumerate}

	The ACS also monitors the system execution. It is responsible for
ensuring that the essential servers are active in the system. If it detects
the termination of an essential server, e.g. when a server crashes, the ACS
sends termination orders to every server. When receiving this termination
order, each server saves its data and terminates its execution. Then, the
ACS reboots the system. This allows a system crash recovery.

\subsection {Process Management}

	In Masix, Unix processes are emulated with the help of the Mach
tasks and threads. A process is a task containing a single thread. Mach
provides the internal process management, e.g. scheduling, context switches...

	The process server provides the higher level process management. It
maintains Unix like system informations, e.g. user identifier, and offers
the process system call interface.

\subsection {Signal Management}

	The process server is also in charge of emulating Unix like signals.
In order to manage the synchronous signals, mapped as Mach exceptions, it
holds the reception right on the threads exceptions ports.

	When a signal, either synchronous or asynchronous, is sent to a
process, the process server saves the thread state and sets up its stack
to call the signal handler.

\subsection {Memory Management}

	Masix contains servers that deal with memory management. These
servers are external pagers and interact with Mach.

	Masix contains two kinds of pagers:
\begin {enumerate}
\item a data pager, which deals with the swap management. This pager reads
and writes pages belonging to the data and stack segments of the processes.
\item a code pager, which reads code pages. As long as a process does not
modify its code, there is no need to save the code pages in the swap area and
this server can read them from the executable files contained in the file
system.
\end {enumerate}

	The data pager manages swap partitions. It uses memory objects to
communicate with Mach: a memory object is associated with each data and
stack segments. A set of blocks is associated with each memory object.

\subsection {Files Operations}
\label {subsection:files}

	In Masix, we have chosen to integrate the file operations with the
virtual memory subsystem. The files are managed by servers used to
load pages in memory. These servers are external pagers.

	When a process reads or writes data, the I/O operation is converted,
by the library routines, to a memory access. Thus, the process reads or
writes data in memory. If the referenced page is not loaded, Mach detects
a page fault and calls an external pager which is responsible for loading
the page from the file blocks. The pager is also called when a modified
page is evicted by Mach and must be written back to the disk.

	The file servers can receive two kinds of requests:
\begin {enumerate}
\item reads and writes requests sent by Mach,
\item file manipulation requests, e.g deletion of a file, sent by the
client processes.
\end {enumerate}

	Masix is able to manage several file systems structures in a
transparent way. It contains two sorts of file servers:
\begin {enumerate}
\item the Virtual File Server (VFS) receives the file based requests and
calls the appropriate file server,
\item several Physical File Servers (PFS) implement the low-level operations
on files. Each PFS is in charge of managing a certain structure of file
systems, e.g. MS/DOS file system, System V file system, BSD Fast File System...
The operations provided by a PFS are used by the VFS when it needs to accede
to the physical data.
\end {enumerate}

	While the VFS can only be called by client processes, each PFS is also
an external pager. When a process opens a file, a memory object associated with
the file is created and Mach can call the PFS paging operations with the
memory object as an argument.

\section {The Virtual System}

	Masix has been designed to allow the easy extension of the system
by adding or replacing servers. The virtual system is built on top of the
local system by using this possibility.

\subsection {Distributed features}

	Masix includes distributed extensions. These features allow a
transparent sharing of resources on a network of stations running Masix. This
resource sharing is implemented by servers running on top of the local
system. These servers communicate by using a network communication layer.

	Resources sharing concerns:
\begin {itemize}
\item processors: process placement and migration allows us to balance the
load on the network,
\item files: distributed file systems allows processes to accede to remote
files in a transparent way on every station,
\item devices: in a network of stations, some devices are often in
limited number. Thus, the system allows processes running on stations to
accede to remote devices. This access cannot be done in a transparent way since
the user has to know the station connected to the device,
\item system informations: numerous system informations, e.g. user
informations, must be shared between the stations.
\end {itemize}

\subsection {Network Communication}
\label {subsection:comm}

	Masix contains a networking layer which is in charge of maintaining
the list of the stations connected to the network. This layer is also
responsible for maintaining the list of remote servers used on the local
station.

	Masix uses local ports as identifiers of remote objects and servers.
Each port related to a remote server is managed by a local interface server
which forwards the requests that it receives to the remote server.

	This way, client processes use an uniform communication scheme
wether the server is local or remote. Both types of servers use Mach
port for receiving the requests.

\subsection {Distributed Files}

	In Masix, files operations are integrated in the virtual memory
subsystem as explained in Section~\ref {subsection:files}. Adding a new file
system type consists in starting a new PFS, which is responsible for
managing the files.

	Remote files are managed by a local PFS. The physical file access is
done by the PFS by forwarding the I/O requests to a distant file server (DFS),
which runs on the station that owns the file system on its local disks. The
DFS is in charge of physical I/O on files: it replies to the requests sent
by the PFSs on the client stations.

	The DFS is also responsible for maintaining the files consistency.
Since several processes running on different stations can accede to the same
files, the system must ensure that the latest version is always used. The DFS
maintains a table of open files and, for each file, the list of blocks sent
to remote stations. When it receives a block read request, the DFS defers
the reply if the block has already been transmitted to a process which holds
write access to the file or if the request concerns a file open in write mode
and the block has already been transmitted to another process.

	Actually, the DFS uses a token mechanism to protect the file access. It
manages a token for each file block. This token can be distributed to
several processes if the file is open in read-only mode but only one
token can be sent if the file is open in write mode.

	Since the DFS maintains a state concerning the open files, a recovery
algorithm is needed in case of a crash. Each PFS also maintains the list of
the tokens. If the DFS crashes and restarts its execution, it broadcasts a
request to every PFS and rebuilds its state from their replies.

\subsection {Access to remote devices}

	In Masix, devices are referenced by Mach ports. A server is in charge
of implementing the physical I/O on a device when it receives I/O requests
from a process.

	When a process wishes to accede to a remote device, it sends a request
on its associated local port. The local server forwards the request to the
remote station connected to the device. This scheme is very similar to the
communication mechanism explained in Section~\ref {subsection:comm}.

\section {Conclusion}

	We have described the Masix operating system. By using the
multi-servers approach, Masix has been designed as a very evolutive and easy
to maintain system. The system contains lots of optimizations needed to
reduce the overhead due to the context switches and messages passing related
to the multi-servers architecture.

	We are currently implementing the system. This first implementation
is done on 386 PC but we plan to port the system on other platforms since it
depends only on a Mach port.

	\bibliography {abbrev,amoeba,dos,chorus,mach,masix,osf1,posix,reseaux,systemes,unix,v}

\end {document}
