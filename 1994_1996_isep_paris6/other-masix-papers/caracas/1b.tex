\subsection {Distributed Tasks Management}

% texte de l'introduction
%

A dynamic task management (DTM) allows applications and users to benefit of all the 
processing power over the network and to have a better reliability on the system. 
To achieve this, each host needs a view of the system's global state.
Hosts also need to take distributed agreement about their choices to improve system performance. 
Last, they need an efficient process transfert mechanism.

The basic characteristics of our DTM are a full transparency to users, migration of tasks for 
load balancing and a fault tolerance objective. 
Our DTM provides a set of services like information exchange between hosts
(e.g. load, state) or
naming and location transparency of entities over the network.

Processes in a system are never executing alone. 
They interact with other processes that build up an application and with the resources managed by the system. 
Since the migration of a process may influence on the performance of the others, we have defined processes group,
which are based on relations between processes.

With non-preemptive placement, processes groups are easy to define with statistics obtained during previous executions or with 
hints provided by the user. 
On the contrary dynamic processes group definition is hard to obtain because hosts only have a partial view of the system. 
Definition of processes group for load distribution induce the definition of forces between processes 
(attraction-repulsion \cite {folliot:these,tokoro90}).

We define different groups of processes, based on memory and CPU usage, communications and devices access. 
Each group has a major gravity centre and fixed points. Distance between remote processes being in the same group
can be evaluated. This distance represents the ratio cost/advantage of migrating them on another host.

%Three major issues have to been solved:
%\begin{itemize}
%        \item Which process shall migrate ?
%        \item When should migration append ?
%        \item Where should execution restart ?
%\end{itemize}

We extend the multi-criteria algorithms defined in \cite {folliot:these} to a dynamic system. 
We use the same refinement technics by using filters linked to the bottlenecks of each host. 
Static constraints (e.g. the binary architecture) allows us to first define the migreable processes. Thru 
the refinements each host computes a set of eligible processes. After a distributed agreement, the group of processes is migrated.
The order in refinements is defined by the resources that will better improve the overall host performances.

	Our migration mechanism concepts are :
\begin{itemize}
        \item We define four migration states depending on the local load (acceptance, inactive, placement, migration).
        \item There are two migration axes. For the hosts based on multicriteria load (i.e cpu, memory, network, processes queue)
 and for each process based on multicriteria forces (i.e. memory usage, cpu utilization and system calls forwarded to the Mach microkernel, 
local or remote servers).
        \item Bottlenecks on each host define the refinements order in migreable process selection.
        \item Local set of inactive hosts is managed. This set is used for finding a destination host. 
The hosts belonging to this set send their load to the others. Each host can enter the set if its load is lower than the last one in the set. 
\end{itemize}

The cost is the sum of the migration mechanism and the overhead due to original host servers access.
The advantage is the performance speed-up for the migrating process and for the local process. 

Knowing host load is not enough to ensure an efficient migration policy. Informations on the network's 
use are needed\cite {dejan93}. The gain for the process can be evaluated in term of number and size of messages exchanged and by the
time the process has waited for replies from the remote servers. 
For each communicating object and for different periods of time, we maintain the communications statistics.
By knowing which hosts are concerned by this objects we can evaluate the most interesting host for
migrating a process in a local point of view.

