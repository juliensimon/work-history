
\documentstyle [a4,epsf,rcnamed] {article}

\parskip=2mm

\begin {document}

\begin{center}
{\Large Overview of the Masix Distributed Operating System}

Franck M\'evel and Julien Simon\\
Laboratoire MASI, Institut Blaise Pascal\\ 
Universit\'e Paris VI\\
4 place Jussieu\\
75252 PARIS CEDEX 05, FRANCE\\
T\'el: (1) 44 27 71 04\\
Fax: (1) 44 27 62 86\\
\{Franck.Mevel,Julien.Simon\}@masi.ibp.fr\\
http://www-masi.ibp.fr/masix

\end{center}

\newcommand {\figps} [4]
{
    \begin{figure} [ht] 
       \begin{center} 
                \leavevmode
                \epsfxsize=#2
                \epsfysize=#3
                \epsffile {#1.ps}
                \caption {#4}
                \label {fig:#1}
        \end{center}
    \end{figure} 
}       

\newcommand {\mysection} [1]
{
        \vspace {-5mm}
        \section {#1}
        \vspace {-3mm}
}

\newcommand {\mysubsection} [1]
{
        \vspace {-1mm}
        \subsection {#1}
        \vspace {-3mm}
}

\newcommand {\myitem}
{
        \vspace {-2mm}
        \item
}


\mysection {Introduction}

A current trend in operating systems research is to use microkernels as a basis
for new systems. 
Indeed, microkernels offer many advantages compared with monolithic systems, 
such as portability, modularity and dynamicity.

Mach has been widely used to build many Unix-like systems, like
Lites~\cite{Helander94} and Mach-US~\cite{mach-us95}. Other personalities
have been implemented on top of Mach, like DOS~\cite{dos91} or 
OS/2~\cite{Phelan93}.
Nevertheless, to the best of our knowledge, none of the operating systems 
built on top of a microkernel has really investigated the issues related 
to multiple personality support.

Masix~\cite{caracas94} is a distributed operating system based on the 
Mach microkernel~\cite{mach:foundation}, currently under development at the 
MASI Laboratory.
The goal of this project is to investigate extensively the problems related 
to multiple personality support. 
We are particularly interested in providing transparent
distributed mechanisms to the user processes, such as load distribution~\cite{paris95}, 
fault tolerance or distributed files. 
For this purpose, we have designed a Distributed Generic Layer~\cite{athens95}
, made of several servers, interposed between the microkernel and 
the personalities.

In this paper, after a brief overview of Mach, we present the design 
principles of Masix and its overall structure. Then, we focus on
two major services provided by the DGL, i.e. naming and communications.

\mysection {The Mach microkernel} 

The Mach microkernel provides basic resource management mechanisms, such as~:
\begin{itemize}
\myitem task and thread management. They are the implementation basis for
processes. A task is a resource allocation unit, whereas a thread is an
execution unit~;
\myitem local interprocess communication (IPC), according to a client-server model
 based on ports and messages. A port is an unidirectional communication channel,
protected by send and receive rights. A message is a set of typed data, sent to
a port~;
\myitem physical and virtual memory management~;
\myitem physical peripherals management.
\end{itemize}

\mysection {Design Principles}

\mysubsection {System Structure}

We believe that it is worth designing an evolutive system in which
components can be added in a dynamic way. Therefore, we have chosen 
to build Masix following the multi-servers approach. 
Implementing the operating system as a set of communicating user mode 
tasks offers many advantages:
\begin {itemize}
\myitem each server has its own interface, that insures protection and minimal
interactions between the other ones~;
\myitem each server can be written and tested without interactions with the
other ones,
\myitem each server is smaller than a traditional kernel and, thus, easier
to maintain,
\myitem adding a server in the running system is easy if a dynamic service
management has been implemented.
\end {itemize}

Nevertheless, the multi-servers approach has also its drawbacks:
\begin {itemize}
\myitem shared data are difficult to implement because each server has its
own address space,
\myitem the execution of system services often involves several servers and
implies message exchanges and context switches. This leads to lower
performances due to the context switch overhead.
\end {itemize}

Lots of optimizations have been included in the design of the system
to minimize the number of servers involved in the realization of a system call.
Moreover, we plan to use new mechanisms to reduce context switching between
tasks : after intensive testing of systems servers it should be possible to
take advantage of co-location~\cite{Condict93}. This way, the servers
will run in kernel mode and any context switching and memory copying
between the kernel and servers will not occur anymore.

System calls are implemented as remote procedure calls: a process
sends a request message to a server and waits for the reply. 

\mysubsection {Naming and Protection}

In Masix, each server is responsible for providing a set of well-defined
abstractions and semantics. The combination of the whole set of servers
forms the complete operating system.

Each server implements the operations that act on typed objects. The
server implements a fixed dialog protocol related to the
type of the objects. When a client process needs to act on an object, it sends
a request message to the server that manages the object. This message is sent
on a Mach port, which is used as the object identifier. The server holds the
reception right on this port.

Accessing an object only requires to know its
port and its associated dialog protocol. The request sent to the object
port is forwarded by the Mach microkernel to the server which manages the
object. This communication scheme allows several servers, which manage the
same objects types in different ways, to coexist in the system as long as
they implement the same dialog protocol.

Using Mach ports as objects references allows a high level of protection.
To act on an object, a process needs to hold a send right on its port.
When a process accesses an object for the first time, i.e. when it creates or
opens the object, the server returns the send right if the calling
process is allowed to act on the object. Otherwise, it does not return
any send right. Thus, if the server has denied access to the object, the
process cannot send any request since it does not hold any send right to the
port.

\mysubsection {Server Structure}

In Masix, each server is a Mach task. This task contains several
threads which wait for requests and execute them.
Each object is named by a port and every object
port is contained in a single port set. Every thread waits for requests on
this port set. Requests are thus dynamically distributed between the
threads contained in the server.

\mysubsection {Interactions between servers}

Each server needs to maintain some information relative to its
objects. Two servers often need to use the same information. In a monolithic
kernel, these informations are stored in common tables which are accessed by
every component which needs to. In a multi-server system, we cannot use global
tables since each server has its own address space. In Masix, each server
maintains a subset of the system informations. This subset contains only the
informations that the server needs to know in order to manage its objects.

\mysection {Overall Structure}

As shown on Figure~\ref{fig:masix2-us}, Masix is made up of two layers, 
themselves built of several servers~:
\begin{itemize}
\myitem the personalities, which implement the semantics of traditional
operating systems : Unix, DOS, OS/2, Win32,~\dots~;
\myitem underneath, the DGL, which offers generic services to the multiple 
personalities (communication, shared memory, process management, fault 
tolerance,~\dots.
\end{itemize}

\figps{masix2-us}{125mm}{85mm}{Overall structure of the Masix System}  

\section {The Distributed Generic Layer}

This layer consists of a set of communicating servers
which offer distributed services to the personalities. 
We now present two of these servers : the Administration and Configuration 
Server and the Generic Network Server.

\mysubsection {The Administration and Configuration Server}

The Administration and Configuration Server (ACS) is the first server
to run when Masix is started up. It is in charge of maintaining the list of
the servers which run on a node. 
When a server starts its execution, it registers itself with
the ACS by sending a message on its port. This message contains a character
string used as the name of the server, and a send right on its request port.

When a client wishes to communicate with a server, it sends the string to 
the ACS and requests a send right on the corresponding port. Nevertheless, 
not any client should be allowed to request any port in the system.
So, we have designed the following security mechanismss~:
\begin{itemize}
\myitem separation of the inter-server and the client-server communication 
interfaces~;
\myitem authentication of the servers by the ACS~;
        \begin{itemize}
        \myitem when they register, thanks to digital signatures~;
        \myitem when they request a name resolution, thanks to a secret key 
algorithm~;
        \end{itemize}
\myitem mutual authentication of local and remote servers, thanks to a secret key algoritm~;
\end{itemize}

\mysubsection {The Generic Network Server}

Many issues arise because of the modularity of Masix,
linked to the distribution of the global state of the system among 
the microkernel and the various servers. As a consequence, many
communications are initiated between servers running on a same node on
the one hand, and between servers running on different nodes on the other hand.
Thus, Masix needs adequate communication mechanisms to minimize their extra 
cost. 

Our solution is based on a Generic Network Server~\cite{rennes95}, 
interposed between the personalities and the microkernel. 
By using proxy ports and a global name resolution protocol, we scale the 
local Mach IPC up to a workstation network, without altering their 
original semantics
The GNS also provides encryption services, to make up for the insecurity 
of the communication links. This way, eavesdropping can be defeated.

\mysection {Conclusion}

We have presented the structure of the Masix distributed operating
system, as well as some features of its Distributed Generic Layer. 
We briefly explained the naming and communication services it transparently 
provides to the personalities.

\bibliographystyle{rcnamed}
\bibliography {etat,masix,mach,wanted}

\end {document}
