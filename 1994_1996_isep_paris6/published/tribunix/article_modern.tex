% -*-latex-*-
%
% Modernized version of the original article.tex
%
\documentclass[11pt,a4paper]{article}
\usepackage[utf8]{inputenc}
\usepackage[T1]{fontenc}
\usepackage[french]{babel}
\usepackage{geometry}
\usepackage{amsmath}
\usepackage{amsfonts}
\usepackage{amssymb}

\geometry{margin=2.5cm}

\begin{document}

\begin{center}
{
\Large
Le système d'exploitation Linux
}

Remy Card, René Cougnenc, Julien Simon

Remy.Card@linux.org, Rene.Cougnenc@freenix.fr, Julien.Simon@freenix.fr

\end{center}

\section{Historique de Linux}

	Au cours de l'année 1991, un étudiant finlandais, nommé Linus Torvalds,
a acheté un micro-ordinateur de type PC, afin d'étudier la programmation du
microprocesseur i386. Ne voulant pas être limité par MS/DOS, il a tout d'abord
utilisé un clone d'Unix, peu cher, appelé Minix. Minix possède lui-même
certaines limitations qui, bien que moins importantes que celles de MS/DOS,
sont assez gênantes (limitation de la taille des exécutables à 64 kilo-octets,
limitation des systèmes de fichiers à 64 méga-octets, temps de réponse
déplorable, \ldots). Aussi, Linus Torvalds a commencé à ré-écrire certaines
parties du système afin de lui ajouter des fonctionnalités et de le rendre
plus efficace et a diffusé une distribution source de son travail via
Internet. La première version de Linux (version 0.1 en août 1991) était née.

	Cette première version était très limitée (elle ne comportait même pas
de gestionnaire des disquettes) et n'était utilisable que sous Minix. Aussi, il
est fort probable qu'elle ait été utilisée par très peu de personnes.
Néanmoins, un
petit nombre de << hackers >>\footnote{<< hacker >> est ici employé dans son
sens originel, c'est-à-dire une personne compétente passionnée passant le
plus gros de
son temps à coder des programmes utiles, et non dans le sens où il est parfois
employé pour désigner des << pirates >> informatiques.} ont découvert, à cette
époque, cet embryon de système et ont vu en lui la base d'un système
exploitable. Aussi, un certain nombre de personnes ont commencé à travailler
avec Linus Torvalds afin d'enrichir le noyau.

	Au cours des quatre dernières années, le développement du noyau Linux,
ainsi que des utilitaires nécessaires pour en faire un système compatible avec
Unix, a été mené de manière intensive par un ensemble de programmeurs, situés
aux quatre coins du monde, communiquant via le réseau Internet. Ces
développeurs ont implémenté de nombreuses fonctionnalités qui font de Linux
aujourd'hui un clone efficace d'Unix pour micro-ordinateurs PC-386, Amiga
et Atari\footnote{Des portages de Linux sont en cours et le noyau devrait
fonctionner assez prochainement sur stations de travail Sparc et sur PC
Alpha.}.

\section{Méthode de développement}

	La façon dont Linux a été développé (et continue à être développé) est
assez originale. En effet, le développement de Linux n'est pas contrôlé par une
organisation responsable du système~: un ensemble de développeurs, répartis
dans le monde entier, collabore pour étendre les fonctionnalités du système,
le plus souvent en dialoguant via Internet. Tout programmeur disposant des
compétences nécessaires, de temps libre, et d'une certaine dose de courage,
peut participer au développement du système.

	Bien que le nom << Linux >> se réfère au noyau du système, le
développement ne se confine pas à ce seul noyau~: certaines équipes
travaillent sur d'autres projets, comme la conception et le développement
de nouveaux utilitaires ou encore le développement des librairies partagées
utilisées pour programmer.

	Chaque équipe travaille selon une structure hiérarchique informelle~:
une personne est responsable d'un projet et plusieurs autres
programmeurs participent au développement en contribuant du code. L'exemple
typique de cette méthode de développement est le noyau lui-même~: Linus
Torvalds maintient le noyau et c'est lui qui effectue les distributions source
quand il estime que le code est utilisable~; chaque partie importante du
noyau (comme la gestion du réseau, les gestionnaires de périphériques, le
système de fichiers, \ldots) est sous la responsabilité d'un développeur
qui centralise le travail d'autres programmeurs et les transmet à Linus
Torvalds pour inclusion dans le noyau officiel\footnote{Évidemment, le
responsable en question ne se contente pas de coordonner le développement et
programme également.}.

	Bien que cette méthode de développement puisse surprendre au premier
abord, elle est très efficace. La totalité du noyau de Linux a été écrite
par des volontaires, qui ont souvent passé des nuits entières à programmer
et à corriger des bogues.

	Le code développé dans le cadre de Linux est le plus souvent diffusé
sous forme de programme source, sous la licence GNU (<< General Public
License >>, ou GPL). La GPL stipule que les programmes source sont la
propriété de leurs auteurs et qu'ils doivent être distribués sous forme de
source. Cette licence autorise quiconque à revendre ces programmes mais elle
impose que tout utilisateur puisse avoir accès aux programmes source. De plus,
la GPL impose que toute modification de ces programmes est couverte par la
même licence, et donc que les programmes seront toujours librement disponibles.

\section{Fonctionnalités de Linux}

\subsection{Le noyau}

	Linux offre toutes les fonctionnalités d'un clone Unix sur
micro-ordinateurs PC-386. Il fournit un
environnement de travail multi-utilisateurs, plusieurs personnes peuvent
utiliser la machine au même moment, et multi-tâches, chaque utilisateur peut
exécuter plusieurs programmes en parallèle. Le système fonctionne en mode
protégé, exécute du code 32 bits\footnote{contrairement à d'autres systèmes
qui s'exécutent en mode 16 bits et sont donc moins performants\ldots}, et
utilise les mécanismes de protection du processeur pour garantir qu'aucun
processus ne peut perturber l'exécution des autres ou du système lui-même.

	Le noyau implémente les sémantiques Unix~: processus concurrents,
chargement à la demande des
programmes exécutables avec partage de pages et copie en écriture,
pagination, systèmes de fichiers, support des protocoles réseau TCP/IP.

	Il supporte, de plus, la majorité des périphériques existant dans
le monde PC (y compris les cartes sonores) et permet de relire les partitions
MS/DOS, OS/2 et tous les formats standards de CD/ROM.

\subsection{Applications}

	Les librairies de développement dans Linux sont basées sur les
librairies GNU, de la << Free Software Foundation >>. Ces librairies offrent
un haut degré de compatibilité avec les différents << standards >> Unix (Posix,
BSD, System V), ce qui permet de compiler facilement tout type d'application
disponible au niveau source pour Unix. Ces librairies existent sous forme de
bibliothèques partagées, ce qui signifie que le code des fonctions de librairie
n'est chargé qu'une seule fois en mémoire et que les programmes exécutables
sont plus petits en taille sur les disques.

	La plupart des utilitaires standards Unix sont disponibles sous Linux,
aussi bien les commandes de base que des applications plus évoluées, comme
les compilateurs et éditeurs de texte. La plupart de ces utilitaires sont
des programmes GNU,
qui supportent des extensions qu'on ne retrouve pas dans les versions BSD
ou System V de ces programmes, mais qui restent compatibles avec ces
dernières. Certains programmes, notamment les utilitaires réseau, sont des
programmes BSD. En résumé, pratiquement tout programme Unix diffusé sous
forme de source peut être compilé sous Linux et s'exécute parfaitement, grâce
à la compatibilité implémentée dans le noyau et dans les librairies.

	En plus des programmes standards, Linux supporte de << grosses >>
applications. On retrouve l'interface graphique X Window (XFree86 3 basé
sur X11R6), un environnement de développement très complet comprenant toutes
les bibliothèques standard, compilateurs et débogueurs disponibles sous Unix
(C, C++, Objective-C, Smalltalk, Fortran, Pascal, Lisp, Scheme, Ada, gdb,
\ldots). L'utilisateur dispose également d'outils très puissants de formatage
de texte, comme nroff, \TeX, et \LaTeX\footnote{Cet article a d'ailleurs
été rédigé sous Linux avec l'éditeur Emacs sous X-Window puis formaté avec
\LaTeX.}.

\subsection{Compatibilité avec d'autres systèmes}

	Linux n'est pas compatible directement avec les applications
développées pour d'autres systèmes d'exploitation. Afin de permettre aux
utilisateurs de Linux de bénéficier des applications qu'ils possèdent déjà,
que ce soit sous MS/DOS, Windows ou des systèmes Unix commerciaux, des
émulateurs sont en cours de développement et permettent déjà d'exécuter des
applications << étrangères >>.

	L'émulateur MS/DOS utilise le mode virtuel 8086 du processeur i386
pour exécuter des applications DOS. Il implémente les fonctionnalités de
MS/DOS dans un processus et assure ainsi l'interfaçage entre l'application et
le système en émulant les appels système effectués par le programme. À ce
jour, de nombreuses applications fonctionnent correctement sous l'émulateur
MS/DOS et la liste s'allonge tous les jours.

	L'émulateur WINE est assez similaire à WABI, développé par Sun
Microsystems~: il permet d'exécuter des applications Windows en convertissant
leurs appels graphiques en requêtes adressées à l'environnement X Window. À
ce jour, seul un petit nombre d'applications Windows fonctionne correctement
mais le développement de WINE n'en est qu'à ses débuts et les progrès semblent
prometteurs.

	Le module de compatibilité iBCS2 permet d'exécuter des
applications développées pour des systèmes Unix commerciaux sur
micro-ordinateurs PC-386. Cet émulateur convertit les appels système se
conformant au standard iBCS2 (qui définit le format des primitives système
ainsi que celui de leurs arguments) en appels natifs traités par le noyau
Linux. Il est ainsi possible d'exécuter de manière transparente des programmes
développés pour d'autres systèmes, comme SCO par exemple.

	Le but de ces différents émulateurs est de permettre d'utiliser des
applications commerciales sous Linux. Il faut désormais signaler que certains
éditeurs de logiciels considèrent maintenant Linux comme un marché potentiel
pour leurs produits et envisagent de porter leurs applications sous Linux. De
la sorte, il est probable qu'un certain nombre d'applications commerciales
tourneront bientôt en mode natif sous Linux, sans nécessiter
d'émulateur. L'exemple le plus frappant de cette tendance consiste en

\end{document} 