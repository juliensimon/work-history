\documentstyle [IEEEpaper,epsf,graphicx] {article}

\setpapersizeUSletter

\IEEEtitle{Distributed communication services\\in the Masix system}

\IEEEauthors{
Franck M\'evel and Julien Simon\\
Laboratoire MASI, Institut Blaise Pascal\\ 
Universit\'e Paris VI\\
4 place Jussieu\\
75252 PARIS CEDEX 05, FRANCE\\
T\'el: (1) 44 27 71 04\\
Fax: (1) 44 27 62 86\\
\{Franck.Mevel,Julien.Simon\}@masi.ibp.fr\\
http://www-masi.ibp.fr/masix
}


\newcommand {\figps} [4]
{
    \begin{figure} [ht]
       \begin{center}
                \includegraphics[width=#2,height=#3,keepaspectratio]{#1.ps}
        	\caption {#4}
        	\label {fig:#1}
        \end{center}
    \end{figure}
}


\begin{document}

\IEEEheader

\section*{Abstract}

	A current trend in operating system research consists in implementing
	the system in user mode with a minimal kernel. This kernel, called a
	microkernel, provides resource management and some elementary 
	services, e.g. physical memory management or process management.
	Many operating systems, both centralized and distributed, 
	have been built on top of a microkernel.
	Nevertheless, none of them has really investigated the problems 
	related to multiple personality support, that is, running 
	simultaneously on the same workstation applications from worlds 
	as different as Unix, DOS, or OS/2.

        Masix is a distributed multi-server operating 
	system based on the Mach microkernel, with multiple personality 
	support. 
        Its main feature is a {\it Distributed Generic Layer} (DGL), 
        which offers distributed services to the personalities. 
        The distributed multi-server architecture of Masix  
        grants it a high modularity, but also raises many issues, 
	such as transparency, security and performance, which 
        cannot be solved without adequate communication services. 

	To provide total 
	transparency, we extend the traditional Mach communication model 
	to a workstation network by interposing a {\it Generic Network Server} 		(GNS) between the tasks and the microkernel. We defined a global 
	name service, based on a name resolution protocol, which allows any
	pair of Mach remote tasks to communicate transparently, using the local
	Mach IPC semantics. Our name server also provides local and remote 
	authentication mechanisms, based on digital signatures and a secret key 
	algorithm.
	To prevent eavesdropping, all remote communications are transparently 
	encrypted by the GNS, using a public key algorithm.
	These security measures can be easily merged into the name service, 
	to yield a secure distributed name resolution protocol. 

	Microkernel based-systems are traditionnaly criticized for their 
	relatively poor performance. As far as network services are concerned,
	experiments show that a good performance level can be reached, provided 	that the distinctive features of microkernels are taken into account.

\section{Introduction}

In the past few years, the design of distributed operating systems has been
moving towards a microkernel approach. 
Indeed, microkernels offer many advantages compared with monolithic systems, 
such as portability, modularity and dynamicity.
However, microkernel-based systems raise new issues, related to the scattering 
of information. To solve them efficiently, the operating system must have 
appropriate ditributed communication services at its disposal.

Masix~\cite{masix:osf} is a distributed operating system based on the 
Mach microkernel~\cite{mach:foundation}, currently under development at the 
MASI Laboratory. After a quick overview of the communication services 
available in Mach, of the structure of Masix and of related work, we 
study some of the requirements of the Masix system as far as communications 
are concerned, i.e. transparency, security and good performance. Then, we 
present the solutions we have designed.

\section {The Mach microkernel}

Masix is based on the Mach 3.0 microkernel, designed at Carnegie Mellon 
University. 
Mach has been widely used to build many Unix-like systems, centralized like
Lites~\cite{Helander94} and Mach-US~\cite{mach-us95}, or distributed
such as Sprite~\cite{Kupfer93}. Other operating system personalities have
been implemented on top of Mach, like DOS~\cite{dos91} or OS/2~\cite{Phelan93}.
As far as we know, no distributed system has investigated the problems
related to multiple personality support.

Mach offers powerful inter-task communication mechanisms~\cite{Draves90},
based on ports. A port is a queue, in which messages can be added or removed.
Actions on ports are carried out via rights, which are granted to the tasks.
A task can own three types of rights on a port~:
\begin {enumerate}
\item a receive right, granted to a single task~;
\item a send right, granted to several tasks~;
\item a send-once right, allowing the task to send one --- and one only --- 
message on the port.
\end {enumerate}

Each port is managed by a single task, which owns the receive right on the
port. Some privileged ports are managed directly by the kernel itself.
Ports are accessed by the tasks via port names (numeric identifiers), which are
internally converted into rights by the kernel. 
A task can send or receive messages on a port. A message is a structure 
containing data, made of~:
\begin {itemize}
\item a header, describing the message~: size, name of the source
 port, name of the destination port, message type,~\dots~;
\item a set of typed data~: type, number of objects, value of the objects. 
\end {itemize}

\section{Structure of the Masix system}

The primary goal of Masix is the simultaneous
execution of multiple personalities, in order to run concurrently on 
a same workstation applications from the Unix, DOS, OS/2 and Win32 worlds.
Masix also pools the resources of a workstation local area network (LAN), 
independently from the personalities that run on each node, thanks to 
a Distributed Generic Layer (DGL).

Masix is built according to the multi-server model, which grants it a high
modularity. Indeed, each server can be written and tested independently from
the others. Also, it is possible to add or remove dynamically a server to
the system.
As shown on Figure~\ref{fig:masix2-us}, Masix is made up of two layers, 
themselves built of several servers~:
\begin{itemize}
\item the personalities, which implement the semantics of traditional
operating systems : Unix, DOS, OS/2, Win32,~\dots~;
\item underneath, the DGL, which offers generic services to the multiple 
personalities (communication, shared memory, process management, fault 
tolerance,~\dots.
\end{itemize}

\figps{masix2-us}{90mm}{70mm}{Structure of the Masix system}

\section{Requirements}

Many issues arise because of the modularity of Masix,
linked to the distribution of the global state among 
the microkernel and the various servers. As a consequence, many
communications are initiated between servers running on a same node on
the one hand, and between servers running on different nodes on the other hand.

We study three of these issues~:
\begin{itemize}
\item transparency,
\item security,
\item performance.
\end{itemize}

\subsection{Transparency}

To simplify the design and implementation of Masix, it is desireable that
remote IPCs follow the same semantics as the local IPCs provided by Mach.
Thus, inter-task communications are based on the {\bf mach\_msg()} primitive,
no matter where the tasks are located.
This way, two communicating tasks do not have to make assumptions about their 
respective locations. 
This is all the more important since some mechanisms of Masix such as 
migration, presented in~\cite{athens95}, can move a running task from one 
node to another.

\subsection{Security}

In Masix, traditional network security issues take a new dimension.
The most important are eavesdropping, message tampering, replaying and
masquerading.

\paragraph{Eavesdropping} 
Communication links are vulnerable to physical probes, 
which listen to the network traffic. If such an attack was successful, the
 confidentiality of inter-node communication would be compromised. It is 
extremely difficult to prevent these attacks because the network can hardly be 
physically secure. Nevertheless, they can be defeated by systematically 
encrypting every message sent on the network. This way, even if a message is 
eavesdropped, its content is unintellegible~; 

\paragraph{Message tampering} 
An ill-disposed or buggy task could intercept 
a message, alter it and send it again. For example, it could change its 
destination, or modify the data it contains. However, if the messages 
are properly encrypted, they cannot be altered without the destination task 
noticing it. Furthermore, this type of attack seems difficult to set up on 
an broadcast type of network, such as Ethernet, and is probably better 
suited to a store-and-forward type of network~;

\paragraph{Replaying old messages} 
A spy could replay a sequence of messages that 
it eavesdropped, which would alter the state of the system. This attack cannot
 be defeated by encryption alone, since the attacker does not have to understand the messages. As a consequence, timestamps must be used to prevent an old 
message from being processed again~;

\paragraph{Masquerading}
A process could gain access to otherwise protected 
information by usurping the identity of an authorized process. Thus, a server 
could be lured by a fake client, which would use its services without 
authorization, and consequently alter its state. In the same manner, a client 
could be lured by a fake server, and disclose confidential information.
As a consequence, communicating entities must mutually authenticate
themselves before any message exchange takes place.

\subsection{Performance}

Most measures indicate that microkernel-based systems have worse networking
performance than monolithic systems~\cite{Maeda92}. One might then conclude 
that microkernels are inherently unfit for distributed architectures.
Yet, most of the systems based on the Mach microkernel rely on unappropriate
 networking code, because it has been written for monolithic systems. For example,
the CMU single-server system, Unix Server~\cite{Golub90}, runs code written 
for 4.3BSD.
As we shall see later, a microkernel-based system can match the 
networking performance of a 
monolithic system, provided that the distinctive features of the microkernel 
are taken into account.

\section {Related work}

We present two major projects related to Mach network IPC~: the netmsgserver and NORMA. 

\subsection{Netmsgserver}

The netmsgserver, designed at Carnegie Mellon University~\cite{Sansom86}, was 
the first project whose goal was to transparently extend the Mach local IPC 
to a workstation network.

It is made of a multi-threaded Mach task, running in user mode on every node
 of the network. The netmsgservers exchange information, so that each of 
them may have a coherent view of the tasks running on every node.

The main services provided by the netmsgserver are~:
\begin{itemize}
\item port translation, thanks to a database containing the following data~:
	\begin{itemize}
	\item a local port, representing a local task~;
	\item a network port, tagged by a unique identifier~;
	\item information related to the security of the communication.
	\end {itemize}
\item port management, i.e. checking the validity of the database, and updating
 it if necessary~;
\item transport protocol management, i.e. segmentation and reassembly, 
flow control, error detection and recovery~;
\item message conversion~: some information included in traditional Mach 
messages is meaningless to a remote task, such as port rights. Thus, it has 
to be converted before the message is transmitted, then converted back when it
 is received. Converting the data included in the message is also mandatory 
when the two communicating entities do not share the same data internal 
representation (little-endian or big-endian)~;
\item name resolution~;
\item message encryption, with an algorithm derived from DES~\cite{des77}~;
\end {itemize}

Thus, the netmsgserver offers transparent and secure communication services 
to Mach tasks. Unfortunately, its performance level is moderate, because each
 message operation requires costly context switches between the task, the 
server and the kernel.

\subsection{NORMA}

NORMA, designed by the Open Software Foudation~\cite{norma93,norma94}, 
is an extension of the Mach microkernel, which allows remote tasks to 
communicate with the semantics of traditional local Mach IPC. This way, 
remote communication is totally transparent.

When a task gives a send right on one of its ports to a remote task, NORMA 
takes charge of this port : it becomes thus a NORMA port. Each one of them 
is tagged by a unique identifier. Each node manages a lookup table, containing 
the NORMA ports it is aware of.

NORMA offers transparent communication services, but requires extensive 
modifications to the microkernel. Howewer, to the best of our knowledge, it
 doesn't include authentication or encryption services. Thus, it is 
probably better suited to tightly-coupled multiprocessor computers than to
 workstation networks.

\section {The Generic Network Server}

The Generic Network Server (GNS) offers communication services capable of 
suiting the requirements of every component of the Masix system, i.e. the 
generic servers and the personality servers.

\subsection{Transparency}

To achieve total transparency, we extend the traditional Mach 
communication model by interposing the GNS between the tasks and 
the microkernel.

The first step of every communication, local or remote, is name resolution. 
In Masix, name resolution is provided by the Administration and 
Configuration Server (ACS).
Our name resolution protocol enables a task 
to request from its local ACS the name of every task running on the
 network.

\paragraph{Name resolution protocol}

Let us consider two tasks A and B, running on different nodes,  
registered with their respective name servers. 
We assume that neither A nor B has ever communicated with a remote task.
We call \(ACS_{a}\), \(ACS_{b}\) the name servers and \(GNS_{a}\), \(GNS_{b}\) the 
network servers.
The protocol works as shown on figure~\ref{fig:name_resolve}~:

\figps{name_resolve}{90mm}{55mm}{Name resolution between two remote tasks}

\begin{enumerate}
\item A sends B's name to \(ACS_{a}\), and requests the corresponding port~;
\item since B is a remote task, \(ACS_{a}\) doesn't know this port. So, it 
requests the resolution of B's name from \(GNS_{a}\)~;
\item \(GNS_{a}\) broadcasts a name resolution request, containing the name of 
the wanted task, as well the node number of the request's originator~;
\item \(GNS_{b}\) gets a copy of the request. Since B has never communicated with
a remote task, it is unknown to \(GNS_{b}\). So, \(GNS_{b}\) sends B's name to 
\(ACS_{b}\) and requests the corresponding port~;
\item \(ACS_{b}\) sends B's port to \(GNS_{b}\)~;
\item \(GNS_{b}\) stores this information for further use, and answers to 
the request of \(GNS_{a}\) by sending it B's port and its node number. The pair 
{\it (node number,port)} constitutes a global identifier, thanks to which 
a task can be located.
\item \(GNS_{a}\) stores this identifier, and creates a proxy port for B, 
giving itself the receive right. Then, it sends the proxy's name to 
\(ACS_{a}\), and gives it a send right on this port~;
\item \(ACS_{a}\) sends the proxy's name to A, and gives it a send right on 
this port. As soon as A gets this information, it can communicate with B, 
as shown on figure~\ref{fig:a-b}~;

\figps{a-b}{80mm}{55mm}{Communication between two remote tasks}

\item A sends a message on B's proxy port~;
\item since \(GNS_{a}\) owns the receive right on this port, it receives the
 message. It translates the local port names (B's proxy port and A's reply port) into global identifiers, and sends the message to \(GNS_{b}\)~;
\item \(GNS_{b}\) accomplishes the inverse translation and delivers the message 
to B.
\end{enumerate}

\paragraph{Optimizations}

This name resolution protocol may seem extremely time-consuming. However, the 
full resolution of B's name only takes place once on A's node. Let us suppose 
that another task running on A's node wishes to communicate with B. It asks 
\(ACS_{a}\) for B's port (step 1). Since this information is cached by \(ACS_{a}\),
 the task gets an immediate answer (step 8).
Anyhow, if a task located on a third node wishes to communicate with B, 
the full resolution of B's name must occur on that node. 
Nevertheless, further resolutions of B's name can be prevented by modifying 
step 6. 
Instead of replying only to \(GNS_{a}\), \(GNS_{b}\) can broadcast B's unique 
identifier to every GNS on the network. Thus, B will be known by every node, 
and the resolution of its name will only consist in a call to 
the local GNS.

What happens if B wants to reply to A ? First of all, A's name must be 
resolved in the same way B's name has been. Again, this full resolution can
be prevented by modifying step 3. Instead of sending A's global 
identifier only to \(GNS_{b}\), \(GNS_{a}\) broadcast it to to every GNS on the network. This way, A will also be known by every node.

These optimisations induce extra remote communication between the nodes.
However, insofar as unicasts are replaced by broadcasts, the number of messages
does not increase. As a consequence, we believe that the cost of this extra 
communication will be very low. We are currently implementing this protocol, 
and extensive performance measurements will determine its efficiency.

\subsection{Security}

Our goal is to provide the environments and the DGL with services allowing 
them to defeat the attacks we mentioned earlier, or at the very least to 
complicate their implementation.

Our approach is based on the following principles~:
\begin{itemize}
\item separation of the inter-server and the client-server communication 
interfaces~;
\item encryption of every message sent between the network servers, thanks to
a public-key algorithm~\cite{Diffie76}~;
\item authentication of the servers by the ACS~;
	\begin{itemize}
	\item when they register, thanks to digital signatures~;
	\item when they request a name resolution, thanks to a secret key 
algorithm~;
	\end{itemize}
\item mutual authentication of the servers, thanks to a secret key algoritm~;
\end{itemize}

\subsubsection{Communication interface separation}

Each server has two communication interfaces~:
\begin{itemize}
\item one through which it receives the clients' requests. Its ports are 
available on request from the ACS~;
\item another one through which it communicates with other servers. Its ports 
are also available on request from the ACS, but their delivery is
submitted to authentication. Prior to any port delivery, the ACS 
makes sure that the other party is a legitimate server.
\end{itemize}

This separation protects the ports used by inter-server communication, and
prevents a rogue process from acquiring a send right on one of them.

\subsubsection{Message encryption}

Encryption in Masix is based on a public key algorithm. This method 
does not require any beforehand knowledge of a secret key shared by 
the communicating entities. This is why it is well suited to distributed 
architectures.

\paragraph{Principle}

The principle of such an algorithm is very simple. Each communicating entity 
is attributed two keys : a public key and a secret key. Both are 
generated by a mathematic algorithm, such as RSA~\cite{Rivest78} or 
LUC~\cite{Smith92}, which guarantees that it is extremely difficult to compute 
the secret key from the public key.
The public key of an entity is used to encrypt the data sent to this entity.
So, before being able to send a message, one must obtain the public key of 
the addressee. 
When the addresse receives the message, it decrypts it with its secret key,
which must be unknown to anybody but its owner.

\paragraph{Public key distribution}

Each GNS knows the public key of all the others. Indeed, when a GNS starts,
it broadcasts its public key to its homologues, and requests that they send it
theirs. This exchange must of course be encrypted and authenticated, 
to prevent a user 
process from masquerading as a GNS. One solution to this problem could be 
to encrypt the request with the super-user's key, whose ownership acts as
an authentication.

\paragraph{Encryption}

Let us consider two tasks A and B, which want to communicate in a secure 
manner. We note \(C_{K}(...)\) a message encrypted with key K.

A secure communication takes place as follows~: 
\begin{enumerate}
\item A sends a message on B's proxy port~;
\item \(GNS_{a}\) receives the message, because it has the receive right on 
this port. It then encrypts the message with the public key of \(GNS_{b}\), \(pub_{{R}_{b}}\), and sends it to \(GNS_{b}\)~;
\item only \(GNS_{b}\) can decrypt it, thanks to its secret key, 
\(sec_{{R}_{b}}\). Once this is done, it delivers the message to B~;
\item when B replies to A, \(GNS_{b}\) encrypts the message with 
\(pub_{{R}_{a}}\), and sends it to \(GNS_{a}\)~;
\item only \(GNS_{a}\) can decrypt it, thanks to \(sec_{{R}_{a}}\). 
Once this is done, it delivers the message to A~;
\item and so on~\dots
\end{enumerate}

\subsubsection{Server authentication by the ACS}

At start-up, the personality servers and the DGL servers register with 
the ACS. However, a user process must at all costs be prevented from
doing so. If it succeeded, it could request the ports of all the "private"
server ports, which would allow it to obtain privileged information, or to
alter the system's state.

When it wishes to register, a server must send a port to the ACS, as
well as the name under which the port should be registered. The ACS 
asks the process server to find the owner of the port, and to compute the
digital signature of the owner's binary. This computation could for instance
be based on the MD5 algorithm~\cite{Rivest92}.

If this signature is not listed in the signatures authorized by 
the ACS, this means that the process which tried to register is not a
legitimate server (the list of signatures could for instance be stored as 
a disk file, encrypted with the super-user's key). In this case, the 
registration fails.

If the signature is listed, the registration succeeds, and the ACS sends
back to the server a signature, which now acts as a secret key. The ownership
of this key proves that its holder is a legitimate server. Thanks to this key,
which must be provided at each name resolution request, the ACS can make sure
that its party is a registered server and not a impostor.

\subsubsection{Authentication of local servers}

We use a secret key authentication protocol, inspired by~\cite{Needham78}.
The signature provided by the ACS when a server registers acts as the secret 
key of the server.
Authentication between two servers A and B, illustrated in 
figure~\ref{fig:key-local}, takes place as follows~:

\figps{key-local}{60mm}{45mm}{Authentication betwen two local servers}

\begin{enumerate}
\item A sends to \(ACS_{a}\) the message (A, B), i.e. A's and B's identities~;
\item \(ACS_{a}\) computes a new secret key, \(sec_{ab}\), based on
\(sec_{a}\) and \(sec_{b}\), which will only be used for communications
between A and B. Then, it sends to A the message
 \((B, sec_{ab}, C_{sec_{b}}(sec_{ab}, A))\)~;
\item A stores \(sec_{ab}\) for further use. Then, it sends to B the message 
\(C_{sec_{b}}(sec_{ab}, A)\)~;
\item only B can decrypt this message and extract \(sec_{ab}\). From now on, 
only A et B have a copy of this key. As a consequence, its ownership 
guarantees the identity of its holder.
\end{enumerate}

The original protocol includes an extra message exchange. Indeed, there is no
guarantee that the message received by B on step 4 comes from A. It could
be an eavesdropped message, replayed by a process which does not have 
\(sec_{ab}\). However, since this is a local communication, it cannot
be eavesdropped.

\subsubsection{Mutual authentication of remote servers}

When A and B run on different nodes, the distribution of the secret key 
 \(sec_{ab}\) is more complicated, as shown on figure~\ref{fig:key-remote}~:

\figps{key-remote}{90mm}{50mm}{Authentication between two remote servers}

\begin{enumerate}
\item A sends to \(ACS_{a}\) the message (A, B)~;
\item \(ACS_{a}\) sends to \(ACS_{b}\) the message \((A,B, sec_{a})\). Since \(ACS_{b}\) is remote, \(GNS_{a}\) receives the message on its proxy port~; 
\item \(GNS_{a}\) sends to \(GNS_{b}\) the message \(C_{pub_{GNS_{b}}}(A, B, sec_{a})\)~;
\item \(GNS_{b}\) decrypts it with \(sec_{GNS_{b}}\), and sends to \(ACS_{b}\) the message \((A, B, sec_{a})\)~;
\item \(ACS_{b}\) computes a new secret key, \(sec_{ab}\), based on
\(sec_{a}\) and \(sec_{b}\), which will only be used for communications
between A and B. Then, it sends to \(ACS_{a}\) the message  \((A, B, sec_{ab}, C_{sec_{b}}(sec_{ab}, A))\). Since \(ACS_{a}\) is remote, \(GNS_{b}\) receives the message on its proxy port~;
\item \(GNS_{b}\) sends to \(GNS_{a}\) the message \(C_{pub_{GNS_{a}}}((A, B, sec_{ab}, C_{sec_{b}}(sec_{ab}, A))\)~;
\item \(GNS_{a}\) decrypts it with \(sec_{GNS_{a}}\), and sends to \(ACS_{a}\) the message \((A, B, sec_{ab}, C_{sec_{b}}(sec_{ab}, A))\)~;
\item \(ACS_{a}\) sends to A \((A, B, sec_{ab}, C_{sec_{b}}(sec_{ab}, A))\)~;
\item A  stores \(sec_{ab}\). Then, it sends to B the message \(C_{sec_{b}}(sec_{ab}, A)\)~;
\item only B can decrypt this message and extract \(sec_{ab}\). From now on,
only A et B have a copy of this key. As a consequence, its ownership
guarantees the identity of its holder.
\end{enumerate}

This authentication protocol is extremely similar to the name resolution
protocol. They can consequently be merged easily, so that localization and
authentication happen simultaneously.

\subsection{Performance}

A microkernel-based system can match, or even exceed, the networking 
 performance of a monolithic system, thanks to the following techniques~:
\begin{itemize}
\item taking into account the distinctive features of the microkernel, such as~:
	\begin{itemize}
	\item mapping the device driver into the applications' address 
space~\cite{Forin91}~;
	\item sharing memory between the network server and the device 
driver~~\cite{Reynolds91}~;
	\end{itemize}
\item improving the internals of the protocol, without modifying its 
original semantics~;
\item preventing the network-related system calls from being processed by a 
Unix emulator, and rewriting network applications so that they directly call Mach primitives.
This can be easily done for traditional applications, such as
 {\bf ftp} or {\bf telnet}, but not for every user application~;
\end{itemize}

\figps{libs}{80mm}{55mm}{Network protocol decomposition}

The most notable optimization has been presented in~\cite{Maeda93}.
Indeed, it enables the system to reach the networking performance of a 
monolithic system. This is achieved by splitting a protocol functionalities 
between the network server and a library mapped into the applications' 
address space. This decomposition is shown on figure~\ref{fig:libs}.

The performance-critical services, such as sending or receiving a message, 
are located in the library. This way, many costly context switches between 
the application and the network server can be prevented. The server only 
contains the services which require its explicit intervention, such as 
opening or closing a connection. We are currently working on this 
decomposition.

\section{Conclusion}

The distributed multi-server architecture of the Masix system requires 
appropriate communication services. They should be transparent, secure and 
performant.
Our solution is based on a Generic Network Server, interposed between the 
personalities and the microkernel. By using proxy ports and a global name 
resolution protocol, we scaled the local Mach IPC up to a workstation network,
 without altering their original semantics.
We also proposed several mechanisms which guarantee the confidentiality and
authenticity of the communications. They rely on secret and public key cryptographic algorithms. In particular, we have presented a distributed 
authentication protocol, which establishes the identity of two remote tasks. 
This protocol can easily be merged with the name resolution protocol.
Lastly, we studied several techniques, which improve the end-to-end 
networking performance. The most notable is based on the decomposition of
 network protocols, so that performance-critical primitives are located in
the applications' address space.

\bibliographystyle{unsrt}
\bibliography{security,group,ft,distributed,abbrev,wanted,etat,masix,net,mach}

\end{document}

