\documentstyle [IEEEpaper,rcnamed] {article}
\IEEEtitle{Services de communication r\'epartis dans le syst\`eme Masix}
\IEEEauthors{
\begin{tabular}{cc}
Julien Simon &  Franck M\'evel\\
Julien.Simon@masi.ibp.fr & Franck.Mevel@masi.ibp.fr\\
\end{tabular}
}

\newcommand{\myitem} {
	\vspace{-2mm}
	\item 
}

\IEEEinstitute{Laboratoire MASI, Institut Blaise Pascal
\\Universit\'e Pierre et Marie Curie\\Paris VI}

\frenchspacing

\begin{document}

\IEEEheader

\sloppy

\section{Introduction}
Depuis quelques ann\'ees, la conception de syst\`emes d'exploitation r\'epartis
s'oriente vers des approches \`a base de micro-noyau. En effet, les
micro-noyaux offrent de nombreux avantages par rapport aux syst\`emes
monolithiques, comme la portabilit\'e, la modularit\'e et la dynamicit\'e.

Cependant, les syst\`emes bas\'es sur un micro-noyau posent de nouveaux probl\`emes, 
li\'es notamment \`a la d\'ecentralisation des informations, qui entraine une 
augmentation des communications entre les diff\'erentes composantes du syst\`eme.
Afin de r\'esoudre ces probl\`emes de mani\`ere efficace, le syst\`eme 
d'exploitation doit disposer de services de 
communication r\'epartis adapt\'es.

Cet article d\'ecrit cet aspect du syst\`eme r\'eparti Masix~\cite{masix:osf}, 
bas\'e sur le micro-noyau Mach~\cite{mach:foundation}, en cours de d\'eveloppement
au laboratoire MASI. 
Son objectif principal est l'ex\'ecution simultan\'ee de plusieurs 
environnements, afin d'utiliser 
parall\`element sur une m\^eme station de travail des applications Unix, DOS, 
OS/2, Win32,~\dots. 

Masix permet \'egalement de mettre en commun les ressources d'un r\'eseau 
(processeurs, m\'emoire, p\'eriph\'eriques, fichiers), 
ind\'ependamment des diff\'erents environnements qui peuvent s'ex\'ecuter
sur chaque station.

Nous pr\'esentons tout d'abord la structure g\'en\'erale du syst\`eme Masix
puis nous \'etudions ses services de communication.

\section{Structure de Masix}
Masix est construit selon le mod\`ele multi-serveurs~\cite{mach-us95}, 
qui lui conf\`ere une 
grande modularit\'e. En effet, chaque serveur, qui offre des fonctionnalit\'es 
orthogonales, peut \^etre \'ecrit et test\'e ind\'ependamment des autres.
De plus, il est possible d'ajouter ou de retirer dynamiquement un serveur 
au syst\`eme.
Masix est compos\'e de deux couches, elles-m\^emes form\'ees de plusieurs 
serveurs~:
\begin{itemize}
\myitem les environnements, qui impl\'ementent les s\'e-mantiques des 
syst\`emes d'exploitation traditionnels~: Unix, DOS, OS/2, Win32,~\dots~;
\myitem en dessous, le syst\`eme d'accueil g\'en\'erique r\'eparti, 
qui factorise les 
services r\'epartis dont ils ont besoin, et notamment les 
services de communication sous la forme d'un serveur r\'eseau.
\end{itemize}

La modularit\'e du syst\`eme Masix pose certains probl\`emes,
comme la gestion de l'\'etat global du syst\`eme. En effet,
les informations qui le composent sont r\'eparties entre le micro-noyau
et les diff\'erents serveurs. De ce fait, de nombreuses communications ont
lieu, d'une part entre les diff\'erents serveurs s'ex\'ecutant sur un noeud, 
et d'autre part
entre les diff\'erents noeuds du r\'eseau. Il est donc important
de disposer de m\'ecanismes adapt\'es afin de mini-miser leur surco\^ut.

Le serveur r\'eseau offre des services de communication susceptibles
de satisfaire les besoins de toutes les composantes du syst\`eme Masix, 
c'est \`a dire des serveurs du syst\`eme d'accueil et 
des serveurs qui composent les diff\'erents environnements.

\subsection {Transparence des communications}

Afin de concevoir le syst\`eme Masix de mani\`ere coh\'erente et d'en 
faciliter le d\'eveloppement, il est pr\'ef\'erable que les 
communications inter-processus distantes s'effectuent selon
les m\^emes s\'emantiques que les communications locales offertes par
Mach.

En effet, deux t\^aches qui souhaitent communiquer ne doivent en aucun cas
pr\'ejuger de leur localisation respective, car certains
m\'ecanismes de Masix, comme la migration~\cite{jrprc}, 
provoquent le d\'eplacement 
d'une t\^ache d'un noeud du r\'eseau vers un autre noeud. 
Par ailleurs,
plusieurs occurences d'un m\^eme
serveur peuvent coexister sur le r\'eseau, afin d'assurer une meilleure
tol\'erance aux fautes. Dans ce cas, une t\^ache ne peut pas savoir
a priori laquelle d'entre elles va traiter sa requ\^ete.

\subsection {Support multi-protocoles}

Le syst\`eme Masix est un syst\`eme multi-environnements. Par cons\'equent,
le serveur r\'eseau g\`ere les protocoles 
propres \`a chaque environnement (TCP/IP, IPX, \dots). 
De plus, il connait \`a tout moment l'\'etat des communications
initi\'ees par les applications qui utilisent cet environnement. 

En effet, la terminaison d'un environnement peut n\'ecessiter la fermeture
de toutes les connexions g\'er\'ees par ses protocoles r\'eseau. Il est donc
important que tous les protocoles utilis\'es par les environnements soient
connus du serveur r\'eseau.
Pour ce faire, celui-ci poss\`ede une interface
d'enregistrement et de d\'esenregistrement dynamiques des
protocoles~\cite{Tschudin91}.

\subsection {Fiabilit\'e des communications}

Un syst\`eme r\'eparti s'ex\'ecutant sur un r\'eseau local est \`a la merci des 
d\'efaillances des noeuds et des liens de communications. Ce probl\`eme
se pose avec d'autant plus d'acuit\'e dans Masix que celui-ci est compos\'e 
de nombreux serveurs, qui peuvent chacun tomber en panne. Par cons\'equent,
un serveur peut \^etre r\'epliqu\'e afin de garantir la continuit\'e de son 
service.

Masix dispose donc de primitives 
de communication de groupe fiables~\cite{Schiper93,Hadzi94} permettant 
d'une part de d\'etecter les d\'efaillances puis d'y rem\'edier, et d'autre part
de pr\'eserver la coh\'erence des donn\'ees r\'epliqu\'ees.

\subsection {S\'ecurit\'e des communications}

Dans Masix, les probl\`emes traditionnels de s\'ecurit\'e des communications r\'eseau
prennent de l'ampleur.
En effet, les serveurs \'echangent par le r\'eseau des informations
priv\'ees, qui ont trait \`a l'\'etat global du syst\`eme. 
Si ces informations sont intercept\'ees ou alt\'er\'ees, 
la s\'ecurit\'e et l'int\'egrit\'e du syst\`eme peuvent s'en trouver gravement
compromises. C'est pourquoi le serveur r\'eseau met \`a la disposition des
environnements des services de chiffrement.

De plus, gr\^ace \`a son interface d'enregistrement, ce serveur
contr\^ole l'acc\`es aux p\'eriph\'eriques r\'eseau, 
puisqu'une application 
ne peut communiquer qu'au travers des protocoles enregistr\'es aupr\`es
du serveur. Ceci emp\^eche une application mal intentionn\'ee 
d'utiliser son propre protocole et d'acc\`eder ainsi librement au r\'eseau.

\subsection {Performances}

La plupart des mesures montrent que les syst\`emes
bas\'es sur un micro-noyau ont des performances r\'eseau inf\'erieures
\`a celles des syst\`emes monolithiques~\cite{Maeda92}. On
pourrait donc en d\'eduire que les micro-noyaux sont intrins\`equement
inadapt\'es aux architectures r\'eparties.

Or,~\cite{Maeda93} montrent qu'il est possible d'atteindre 
les performances d'un syst\`eme monolithique en d\'ecomposant les
fonctionnalit\'es d'un protocole entre le serveur r\'eseau et une librairie projet\'ee 
dans l'espace d'adressage des applications. 

La librairie contient 
les services dont les performances sont critiques, comme l'envoi ou 
la r\'eception d'un message, ce qui permet d'\'eviter de co\^uteux changements 
de contexte entre l'application et le serveur r\'eseau. Ce dernier ne contient
plus quant \`a lui que les services qui n\'ecessitent son intervention, comme
l'ouverture et la fermeture d'une connexion. 

\section {Conclusion}
Les services r\'epartis offerts par le serveur r\'eseau du syst\`eme Masix sont : 
\begin{itemize}
\myitem l'extension des s\'emantiques de communication inter-processus de Mach
\`a l'\'echelle d'un r\'eseau local ;
\myitem la gestion dynamique des protocoles utilis\'es par les environnements ;
\myitem des primitives fiables de communication de groupe (
broadcast, multicast) ;
\myitem l'authentification des entit\'es communicantes et le chiffrement des donn\'ees 
transmises ;
\myitem des performances proches de celles des syst\`emes monolithiques. 
\end{itemize}

\bibliographystyle{rcnamed}
\bibliography{group,wanted,etat,masix,net,mach}
\end{document}
